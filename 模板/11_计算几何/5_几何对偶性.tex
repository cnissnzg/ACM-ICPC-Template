利用对偶性,我们可以将点问题与线问题互相转化。 \\
我们说方程为$y=kx+b$的直线与坐标为$(k,b)$的点是对偶的。 \\

例如典型的半平面交问题,我们可以做如下变换: \\
按半平面的方向将所有的半平面分成两部分,分别是朝$y$的正方向和朝$y$的负方向。 \\
我们对所有方向向上的半平面对偶出来的点取上凸包,对所有方向向下的半平面对偶出来的点取下凸包,则得到的两个半凸包包含了哪些点就意味着半平面交得到的两个半凸包包含了哪些直线。 \\
这对于半平面交的存在性判定非常有帮助。 \\
不过其中平行于y轴的直线会存在问题。我们可以做一些特殊处理:将$x=b$换成$y=\inf x-b\inf$的形式。若方向向左,则认为它是方向向上;若方向向右,则认为它是方向向下。但这样是不可能做到精确的。 \\

我们看一个例题:

现在我们有$n$条垂直于水平线的竖直线段,问能否找到一条直线,使之可以穿过的线段。我们约定,如果某条直线恰好穿过了线段的端点也表示它穿过了这个线段。 \\

可以证明,我们可以把每个线段上端点对偶成向下的半平面,下端点对偶成向上的半平面,转化该问题为半平面交问题,如果半平面交为空则不存在一条直线满足题意。(半平面交为一个点也算是不为空) \\
接着我们再做一次对偶,把两个半平面交问题转化为两个半凸包问题,只要得到的两个半凸包的交为空,则存在这样的直线。(交成一个点也算是为空) \\

但是这种做法只适用于所有的线段平行的情形下。如果是任意的线段,那该问题在对偶为半平面交的问题时,会存在double edge的情况,那么在对偶到凸包问题时是有问题的,不能简单地用$y$高度来决定它是属于上端点还是下端点。类似地我们要把半平面交的“或”的情况加上去,这种“或”的结果就是两个端点的归属是可以互换的。 \\

所以在对于不平行的情况下,问题其实就是在每个线段中选一个端点组成一个集合,另外一个组成一个集合,求两个集合的半凸包,只要存在一个这样的集合方案使得凸包没有交,那么就答案就是存在。这就是一个可分割性的问题了。 \\

\lstinputlisting{"./模板/11_计算几何/5_几何对偶性/ex1.cpp"}
